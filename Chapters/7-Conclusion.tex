\clearpage
\chapter{Conclusion}\label{chap:conclusion}

This study showed that evolutionary computing can be used to evolve players for
the non-deterministic game Monopoly. These players can perform well in the
simulated environment in which they evolved. These players also appear to use
strategies similar to those that have been developed heuristically by real-world
players.

Monopoly is game with a large random component. Every turn brings another roll
of the dice and no game is ever the same. This seemed to analogous to the
research conducted in to games with noise introduced into the
experiments~\cite{Panait02acomparative}. In those experiments, good players were
able to evolve with small populations and relatively small numbers of
generations.

However, what was found in this research is that the random component of
Monopoly introduces a great deal of noise, and makes the search space
significantly larger than for games used in earlier research. So much so that
the smaller populations in this study were not able to evolve players as well as
the larger populations, or as well as populations that played large numbers of
games.

We also found that as much as the decision to buy property is important to
playing the game of Monopoly, property trading is also a significant component.
The first human versus computer validation for the evolved Monopoly players was
based on previous research which indicated that player's should propose trades
often, and propose trades that balanced gains and
losses~\cite{Yasumura2001Negotiate}. 

What we found under actual play conditions is that the AI players did better
when they valued gains much more than losses, and when they used a higher
threshold for accepting a trade that gave them a gain. This seems to contradict
the previous research into trading strategies that the trading algorithm was
based on. Of course, the evolutionary aspect of the players was not involved in
the trading algorithm, so the RGA players were not able to evolve better
strategies during the competition, unlike the human players who undoubtedly
learned as they participated in the competition.

At this point the author thinks that this thesis supports the following:
\begin{itemize} 
  \item {The poor evolution of a simple bit string chromosome (the SGA players) 
  confirms what has been found in other problem domains: the chromosome for any
  evolutionary computing algorithm should be matched to the problem domain.} 
  \item {All the literature cited in this thesis regarding tournament fitness 
  evaluations used simple two-player competitions. This research showed that 
  these competitions could be extended to 4 player games with successful 
  results. The author thinks that this is generalizable to n-player games with
  the appropriate adjustments, but that is only suggested, not supported, by 
  this set of experiments.}
  \item {The improved performance of the RGA players after the property trading 
  strategy was improved shows that in games where one can acquire resources
  directly or trade for resources, both components are important to success.}
  \item {In the face of large amounts of randomness (noise) in a problem domain,
  evolutionary computing can still be robust (given proper evolutionary 
  parameters) to find solutions to the problem domain.}
\end{itemize}