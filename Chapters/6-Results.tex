\clearpage
\chapter{Results}\label{chap:results}

Well over 1 billion games were simulated in the process of evaluating the
chromosomes in this experiment. One thousand generations were evolved for every
combination of population size (32, 128, 512, and 1024), chromosome type (RGA,
SGA, and TGA), number of games played by each chromosome in a generation (7, 25,
50, 100), and fitness evaluator (FINISH\_ORDER, NET\_WORTH, NUM\_MONOPOLIES,
NUM\_PROPERTIES, NUM\_WINS, and TOURNAMENT\footnote{For the tournament fitness
evaluator the number of games was determined by the population size and was not
independent. So for TOURNAMENT, 1000 generations were evolved for every
combination of population size and chromosome type}). Results of the
evolutionary process are presented in the first section of this Chapter.

Results for the RGA chromosome are presented first, and in a large amount
detail. We also look briefly at the results for the SGA chromosomes, even though
players with this chromosome tended to evolve poorly. This is most likely due to
the fact that the data structure used for the SGA chromosomes is not well suited
to the problem domain. Finally, results for the TGA chromosomes are also briefly
discussed, but these players were almost identical to the RGA chromosomes and so
do not provide any additional insight. Because the evolution of SGA and TGA
chromosomes do not provide any useful information, the remainder of the chapter
focuses exclusively on results obtained from the RGA chromosomes.

After the evolutionary process was complete, the best players from each
generation were competed in various combinations to validate the results. 

\begin{enumerate} 
  \item {Intra-population validation}
  \begin{enumerate}
    \item {For a given sequence of RGA populations\footnote{e.g., all
    generations for the population of size 1024, 100 games/generation, original
    fitness evaluator of NET\_WORTH}, select the best player from generation
    250, 500, 750, and 999}
    \item {Compete the 4 players in 50 games}
    \item {Record the fitness scores for FINISH\_ORDER and NUM\_WINS}
    \item {Repeat 30 times}
    \item {Use Student's t-test to compare average fitness to determine if
    players from later generations are better than players from earlier
    generations}
  \end{enumerate}
  \item {Inter-population validation for fitness evaluator}
  \begin{enumerate} 
    \item {From generation 999 of a set of populations differing only by
    original fitness evaluator, select the best player under each fitness
    evaluator (results in 6 players)} 
    \item {Compete the 6 players in 50 games}
    \item {Record the fitness scores for FINISH\_ORDER and NUM\_WINS}
    \item {Repeat 30 times}
    \item {Use Student's t-test to compare average fitness to determine if
    any fitness evaluator was better than the others in evolving players}
  \end {enumerate}
  \item {Real-world validation}
  \begin{enumerate} 
    \item {Select the top 4 players from generations 250, 500, 750, and 999 from
    the population RGA, size 1024, 100 games per generation, FINISH\_ORDER
    fitness evaluator}
    \item {compete the players randomly against human players}
    \item {Record the fitness scores for FINISH\_ORDER}
    \item {Compare average fitness and average net worth among all players}
  \end {enumerate}
\end{enumerate}
First, an intra-population validation was performed where the best players from
different generations within a population were competed against each other.
For example, using the RGA-1024-100-NetW population (a population of RGA players
of size 1024, evolved using 100 games per generation, with the NET\_WORTH
fitness evaluator) select the best player from generations 250, 500, 750, and
999. Compete those 4 players in a set of 100 games and record the cumulative
fitness using FINISH\_ORDER and NUM\_WINS\footnote{Regardless of the original
fitness evaluator used in evolution, the validation experiments use
FINISH\_ORDER and NUM\_WINS to compare players.}. Repeat this process 50 times
without further evolution of the players. We then analyze the player's
performance over the 50 trials to determine if players fitness improved over
time. For most populations, players in later generations were better than
players in earlier generations, which suggests that the genetic algorithm was
successful in evolving players. We will also show that populations that play
more games per generation tended to evolve better players than populations with
less games per generation.

Second, players were also validated with an inter-population validation. That
is, the best players from different populations were competed against each other
in various combinations. For example, given the various RGA-1024-100
(chromosome type-population size-games per generation) populations, select the
best player from each fitness evaluator and from generation 999. This will
result in 6 players which then play 100 games against the other players; 
record the cumulative fitness using FINISH\_ORDER and NUM\_WINS. Repeat this
process 50 times without further evolution of the players. We then analyze the player's
performance over the 50 trials to determine if different fitness evaluators were
better than others. Since the first analysis showed that populations that played
more games per generation tended to evolve better players, only players from the
populations that played 100 games per generation were competed\footnote{Except
for TOURNAMENT in which the most games per generation was 9, for the population
size 1024.}. In this validation, various fitness evaluators were compared to see
which fitness evaluator evolved the best players.

The results from the intra- and inter-population competitions are discussed in
the second section of this Chapter.

In the last section of this Chapter, results from games where RGA players played
against human players are presented. Based on the previous intra- and
inter-population validations performed, we picked 4 players from generations
250, 500, 750, and 999 of a population. These evolved RGA players from
various generations were competed against human players. In general, we found
that though the RGA players competed well against other RGA players, they did
relatively poorly against human players. We believe this is due to the property
trading algorithm that was implemented for this competition.

\section{The Initial Generation in the Evolutionary Process}

In a genetic algorithm which can use an objective fitness function, we expect
that the initial population has fitness values which are uniformly distributed.
In other words, the probability of an individual having average fitness is the
same as the probability of having low fitness or high fitness. As the population
is evolved, average fitness should steadily improve until a plateau is reached,
and the average error should decrease to some minimal value.

The expected results for for a competitive fitness function are different. Since
the first generation is randomly created, it is expected that the players will
be randomly spread in their ability to win games. Some players will win games,
some will lose games, and the average player will win about half the games they
play. So the probability that an individual has average fitness is higher than
the probability that an individual has low or high fitness. The distribution of
fitness scores should follow a distribution similar to the binomial distribution
(See Figure~\ref{figure-binomial}).

\subsection{Results for Initial Generations}

When the initial populations of players was run through a single generation, the
actual results matched the expected results quite closely. 

For example, it can clearly be seen that the fitness distribution looks like a
binomial distribution when the FINISH\_ORDER fitness evaluator is used.
Figure~\ref{figure-RGA-G000-N100-FO-initial_fitness} shows the fitness
distribution for the first generation of the RGA-1024-100-FO population. The
figure shows a binomial shaped distribution which appears to be centered around
the population average of 150\footnote{Because the FINISH\_ORDER evaluator
assigns 1, 2, and 3 points to the players in a game, for a total of 6 points per
game, the average fitness for any single game is \(6/4\) or 1.5. So for \(n\)
games the average fitness is \(n * 1.5\). For a population that plays 100 games
per generation, the fitness distribution will be centered on the
average score of 150.}.

\begin{figure}
\centering
%%----start of first figure----
\begin{minipage}[t]{0.47\linewidth}
\centering
\includegraphics[width=1.0\linewidth]{Figures/binomial.png}
\caption[Binomial Distribution]{The expected fitness distribution for the first
generation of a population that uses a competitive fitness function such as
FINSIH\_ORDER or NUM\_WINS. A peak is expected around the average population
fitness. The distribution is similar to a binomial distribution.}
\label{figure-binomial}
\end{minipage}%
\hspace{0.06\linewidth}%
%%----start of second figure----
\begin{minipage}[t]{0.47\linewidth}
\centering
\includegraphics[width=1.0\linewidth]{Figures/RGA_1024_G000_N100_FO.png}
\caption[RGA Finish Order Fitness Distribution, Initial Generation]{RGA
chromosome, size 1024, 100 games per generation, finish order
fitness evaluator, generation 0.}
\label{figure-RGA-G000-N100-FO-initial_fitness}
\end{minipage}
\end{figure}

An example of the fitness distribution for the NUM\_WINS fitness evaluator is
shown in Figure~\ref{figure-RGA-G000-N100-NW-initial_fitness}. The average
fitness for a single game is \(3/4\) or 0.75; for \(n\) games the average
fitness is \(n * 1.5\). The figure shows a binomial shaped distribution which
appears to be centered around the population average of 75.

\begin{figure}
\centering
%%----start of first figure----
\begin{minipage}[t]{0.47\linewidth}
\centering
\includegraphics[width=1.0\linewidth]{Figures/RGA_1024_G000_N100_NW.png}
\caption[RGA Num Wins Fitness Distribution, Initial Generation]{RGA chromosome,
size 1024, 100 games per generation, number of wins fitness evaluator,
generation 0.}
\label{figure-RGA-G000-N100-NW-initial_fitness}
\end{minipage}%
\hspace{0.06\linewidth}%
%%----start of second figure----
\begin{minipage}[t]{0.47\linewidth}
\centering
\includegraphics[width=1.0\linewidth]{Figures/RGA_1024_G000_N100_NetW.png}
\caption[Historgram of RGA Net Worth Fitness Distribution, Initial
Generation]{RGA chromosome, size 1024, 100 games per generation, net worth
fitness evaluator, generation 0.}
\label{figure-RGA-G000-N100-NetW-initial_fitness}
\end{minipage}
\\[\intextsep]

\begin{minipage}[t]{0.47\linewidth}
\centering
%%----start of third figure----
\includegraphics[width=1.0\linewidth]{Figures/RGA_1024_G000_N100_NM.png}
\caption[RGA Num Monopolies Fitness Distribution, Initial Generation]{RGA
chromosome, size 1024, 100 games per generation, number of monopolies
evaluator, generation 0.}
\label{figure-RGA-G000-N100-NM-initial_fitness}
\end{minipage}%
\hspace{0.06\linewidth}%
%%----start of fourth figure----
\begin{minipage}[t]{0.47\linewidth}
\centering
\includegraphics[width=1.0\linewidth]{Figures/RGA_1024_G000_N100_NP.png}
\caption[RGA Num Properties Fitness Distribution, Initial Generation]{RGA
chromosome, size 1024, 100 games per generation, number of properties fitness
evaluator, generation 0.}
\label{figure-RGA-G000-N100-NP-initial_fitness}

\end{minipage}
\end{figure}

Example fitness distributions for the other fitness evaluators are shown in
Figure~\ref{figure-RGA-G000-N100-NetW-initial_fitness},
Figure~\ref{figure-RGA-G000-N100-NM-initial_fitness}, and
Figure~\ref{figure-RGA-G000-N100-NP-initial_fitness}.

All of the other populations, regardless of population size, fitness evaluator,
or number of games per generation, show similar results for the initial
generation. These fitness distributions are not surprising. Three of the fitness
evaluators are competitive fitness functions (NUM\_MONOPOLIES and
NUM\_PROPERTIES are not directly competitive, since the player with the most
monopolies or properties is not necessarily the winner of the game). The fact
that the results match previous research into competitive fitness functions
shows that the evolutionary approach we have taken appears to be correct.

\subsection{Results for Subsequent Generations}

As a population co-evolves under a competitive fitness function the average
fitness will remain the same as the initial population, and the worst and best
fitness scores should tend to converge towards the average. As the players get
better and the population is more evenly matched, most players will tend to win
half the games they play. This is because competitive fitness is measured by
awarding points to players as they play against each other in games. So, as poor
players are removed from the population, the remaining players will tend to be
near each other in ``ability.'' Thus, the distribution of fitness scores will
become much tighter around the average score, and no single player will be able
to dominate the other players.

For an example of this we look at generation 100 of the RGA-1024-100-FO
population. Figure~\ref{figure-100th_gen_fitness} shows the actual fitness
distribution for this population. As predicted, it can be seen that the mean
remained the same, but the variance has appeared to decrease. Whereas the
fitness scores in generation 0 ranged from 103 to 193, in generation 100 they
ranged from 115 to 186. The distribution around the mean tightened relatively
quickly (it can clearly be seen in generation 100), and then remained fairly
constant over the course of the simulation which was 1000 generations.

\begin{figure}[htp]
\centerline{\includegraphics[width=0.75\columnwidth]{Figures/RGA_1024_G100_N100_FO.png}}
\caption[RGA Fitness Distribution, 100th Generation]{RGA chromosome, generation
100, 100 games per generation, finish order fitness evaluator.}
\label{figure-100th_gen_fitness}
\end{figure}

Additional plots of the distribution of fitness scores can be seen in
Figure~\ref{figure-RGA-250th_gen_fitness},
Figure~\ref{figure-RGA-500th_gen_fitness},
Figure~\ref{figure-RGA-750th_gen_fitness}, and
Figure~\ref{figure-RGA-999th_gen_fitness}. Each of these figures shows the same
population at various points in the evolution process. The size of the
population is 1024 individuals with an RGA chromosome. Each individual played
100 games in each generation (K-Random Opponents with \(K\approx300\). The
fitness evaluator is the FINISH\_ORDER evaluator.

\begin{figure}
\centering
%%----start of first figure----
\begin{minipage}[t]{0.47\linewidth}
\centering
\includegraphics[width=1.0\linewidth]{Figures/RGA_1024_G250_N100_FO.png}
\caption[RGA Fitness Distribution, 250th Generation]{RGA chromosome, size 1024,
100 games per generation, finish order fitness evaluator, generation
250.}
\label{figure-RGA-250th_gen_fitness}
\end{minipage}%
\hspace{0.06\linewidth}%
%%----start of second figure----
\begin{minipage}[t]{0.47\linewidth}
\centering
\includegraphics[width=1.0\linewidth]{Figures/RGA_1024_G500_N100_FO.png}
\caption[RGA Fitness Distribution, 500th Generation]{RGA chromosome, size 1024,
100 games per generation, finish order fitness evaluator, generation
500.}
\label{figure-RGA-500th_gen_fitness}
\end{minipage}

\begin{minipage}[t]{0.47\linewidth}
\centering
%%----start of third figure----
\includegraphics[width=1.0\linewidth]{Figures/RGA_1024_G750_N100_FO.png}
\caption[RGA Fitness Distribution, 750th Generation]{RGA chromosome, size 1024,
100 games per generation, finish order fitness evaluator, generation
750.}
\label{figure-RGA-750th_gen_fitness}
\end{minipage}%
\hspace{0.06\linewidth}%
%%----start of fourth figure----
\begin{minipage}[t]{0.47\linewidth}
\centering
\includegraphics[width=1.0\linewidth]{Figures/RGA_1024_G999_N100_FO.png}
\caption[RGA Fitness Distribution, 999th Generation]{RGA chromosome, size 1024,
100 games per generation, finish order fitness evaluator, generation
999.}
\label{figure-RGA-999th_gen_fitness}
\end{minipage}
\end{figure}

Inspecting Figures~\ref{figure-RGA-250th_gen_fitness} through
\ref{figure-RGA-999th_gen_fitness}, it appears that at some point between the
100th generation and the 250th generation\footnote{Because of the large amount
of data generated by the simulations, data for every generation was not output
by the simulation. Instead fitness data was output and saved for the initial
population, the 100th generation, and every 250th generation. Thus, the point at
which the population average fitness reaches the plateau cannot be calculated.
However, in a coevolutionary environment this is not a problem. A coevolutionary
population can continue to improve even though the fitness appears to plateau.
This will be further discussed later in this Chapter.} that the population
average fitness reaches a plateau. This is also demonstrated by looking at some
of the descriptive statistics for each generation. These are shown in
Table~\ref{table-stats-for-s1024-n100-fo} for the RGA-1024-100-FO population.

\begin{table}[ht]
\begin{center}
\begin{tabular}{ | r || r | r | r | r | r |}
\hline                        
Generation & Min & Max & Average & Variance & Std Dev \\ \hline \hline
0   & 103 & 193 & 150 & 227.43 & 15.08 \\ \hline
100 & 115 & 186 & 150 & 126.81 & 11.26 \\ \hline 
250 & 110 & 179 & 150 & 122.02 & 11.05 \\ \hline
500 & 119 & 189 & 150 & 120.44 & 10.97 \\ \hline
750 & 117 & 184 & 150 & 124.63 & 11.16 \\ \hline
999 & 113 & 188 & 150 & 121.10 & 11.00 \\ \hline
\end{tabular}
\caption[RGA, Finish Order, statistics]{Descriptive statistics for various
generations of a single RGA population, size 1024, 100 games per generation,
Finish Order fitness evaluator.}
\label{table-stats-for-s1024-n100-fo}
\end{center}
\end{table}
Looking at the standard deviation column, the standard deviation is 15.08 in the
initial generation, is down to 11.26 by the 100th generation, and then in the
remaining generations (250, 500, 750, and 999) the standard deviation appears to
fluctuate around 11.05. We did not attempt to determine whether this difference
in variance between generations is statistically significant\footnote{To test
the hypothesis that the standard deviations are not equal between generations
would require running (in this example) a population of size 1024 through at
least 250 generations, and then repeating that 250 generation trial enough
times to get a statistically significant sample. Based on the work performed for
this research, doing this for the all of the populations would have taken
several weeks of processing time. It was decided that a better use of resources
would be to do a comparative test between players of different generations and
populations.}.

Additional population statistics are in
Table~\ref{table-stats-for-s1024-n100-netw} for the RGA-1024-100-NetW
population, Table~\ref{table-stats-for-s1024-n100-nm} for the RGA-1024-100-NM
population, Table~\ref{table-stats-for-s1024-n100-np} for the RGA-1024-100-NP
population, and Table~\ref{table-stats-for-s1024-n100-nw} for the
RGA-1024-100-NW population.

\begin{table}[ht]
\begin{center}
\begin{tabular}{ | r || r | r | r | r | r |}
\hline                        
Generation & Min & Max & Average & Variance & Std Dev \\ \hline \hline
0   & 4728 & 11186 & 7499.99 & 1137919.23 & 1066.73 \\ \hline
250 & 4285 & 10374 & 7499.80 &  649447.67 &  805.88 \\ \hline
500 & 5235 & 10242 & 7500.00 &  665909.43 &  816.03 \\ \hline
750 & 5080 & 10977 & 7500.05 &  723209.97 &  850.42 \\ \hline
999 & 5306 &  9831 & 7499.96 &  649001.83 &  805.61 \\ \hline
\end{tabular}
\caption[RGA, Net Worth, statistics]{Descriptive statistics for various
generations of a single RGA population, size 1024, 100 games per generation, Net
Worth fitness evaluator}
\label{table-stats-for-s1024-n100-netw}
\end{center}
\end{table}

\begin{table}[ht]
\begin{center}
\begin{tabular}{ | r || r | r | r | r | r |}
\hline                        
Generation & Min & Max & Average & Variance & Std Dev \\ \hline \hline
0   &  8 & 209 &  80.42 & 1014.77 & 31.86 \\ \hline
250 & 37 & 226 & 110.88 &  758.92 & 27.55 \\ \hline 
500 & 43 & 199 & 111.20 &  707.89 & 26.61 \\ \hline
750 & 44 & 199 & 113.97 &  732.09 & 27.06 \\ \hline
999 & 17 & 217 & 116.52 &  734.49 & 27.10 \\ \hline
\end{tabular}
\caption[RGA, Number of Monopolies, statistics]{Descriptive statistics for
various generations of a single RGA population, size 1024, 100 games per
generation, Number of Monopolies fitness evaluator}
\label{table-stats-for-s1024-n100-nm}
\end{center}
\end{table}

\begin{table}[ht]
\begin{center}
\begin{tabular}{ | r || r | r | r | r | r |}
\hline                        
Generation & Min & Max & Average & Variance & Std Dev \\ \hline \hline
0   & 309 & 1094 & 698.18 &  10271.47 & 101.35 \\ \hline
250 & 504 &  929 & 697.75 &   5322.08 &  72.95 \\ \hline
500 & 478 &  973 & 697.72 &   6126.09 &  78.27 \\ \hline
750 & 462 &  983 & 697.81 &   5928.88 &  77.00 \\ \hline
999 & 438 & 1007 & 697.79 &   6095.32 &  78.07 \\ \hline
\end{tabular}
\caption[RGA, Number of Properties, statistics]{Descriptive statistics for
various generations of a single RGA population, size 1024, 100 games per
generation, Number of Properties fitness evaluator}
\label{table-stats-for-s1024-n100-np}
\end{center}
\end{table}

\begin{table}[ht]
\begin{center}
\begin{tabular}{ | r || r | r | r | r | r |}
\hline                        
Generation & Min & Max & Average & Variance & Std Dev \\ \hline \hline
0   & 27 & 126 & 75.00 & 286.43 & 16.92 \\ \hline
250 & 36 & 120 & 75.00 & 155.68 & 12.48 \\ \hline
500 & 30 & 117 & 75.00 & 168.21 & 12.97 \\ \hline
750 & 39 & 120 & 75.00 & 160.36 & 12.66 \\ \hline
999 & 30 & 114 & 75.00 & 173.74 & 13.18 \\ \hline
\end{tabular}
\caption[RGA, Number of Wins, statistics]{Descriptive statistics for various
generations of a single RGA population, size 1024, Number of Wins fitness
evaluator, 100 games per generation}
\label{table-stats-for-s1024-n100-nw}
\end{center}
\end{table}

In all of the tables, the average population fitness stays approximately
the same, and the variance in population fitness decreases. Even though it
appears that the variance has plateaued in these examples, players in later
generations can still be improving in fitness. In fact, we show later in this
chapter that players in the last generation are statistically better (i.e., they
win more games) than players in the earliest generations.

This section has focused primarily on the RGA-1024-100 sets of populations. As
will be seen in Section~\ref{6_Validation} this set of populations, and
particularly the RGA-1024-100-FO and RGA-1024-100-NW populations, generally
produced the players that competed best against other evolved players. In
general, though, a similar pattern of statistics is seen in all the other RGA
and TGA populations that were evolved in this study.

The SGA populations,
however, were more varied and are discussed in more detail in Section~\ref{6_SGA}.

\subsection{Genome Changes Over Time}

We now look at the change in a genome between the initial and final
generaitons of a population. Using the RGA-1024-100-FO population, part of
the genome for the best player from generation 0 is shown in
Figure~\ref{figure-genome0} and part of the genome for the best player from
generation 999 is shown in Figure~\ref{figure-genome999}. For easier comparison,
the chromosome values from the chart are also shown in
Table~\ref{table_chromosome}.

\begin{figure}[htp]
\centerline{\includegraphics[width=0.75\columnwidth]{Figures/genome.png}}
\caption[Illustration of Genome, Generation 0]{This chart shows part of the
genome of the fittest player in the first generation of the RGA-1024-100-FO
population. This chart compares the chromosome used when the player already owns
a property in the group (represented by the circle symbol) against the
chromosome used when two other players own a property in the group (the diamond
symbol).}
\label{figure-genome0}
\end{figure}

\begin{figure}[htp]
\centerline{\includegraphics[width=0.75\columnwidth]{Figures/genome.png}}
\caption[Illustration of Genome, Generation 999]{This chart shows part of the
genome of the fittest player in the last generation of the RGA-1024-100-FO
population. This chart compares the chromosome used when the player already owns
a property in the group (represented by the circle symbol) against the
chromosome used when two other players own a property in the group (the diamond
symbol). These two genes show agreement with the heuristic strategy: the player
is more likely to buy a property when there is a chance of gaining a monopoly,
and less likely to buy a property when other players are blocking the group from
being monopolized.}
\label{figure-genome999}
\end{figure}

\begin{table}[ht]
\begin{center}
\begin{tabular}{ | r || r | r | r | r | r |}
\hline                        
 &  &  &  &  &  \\ \hline \hline
 &  &  &  &  &  \\ \hline 
\end{tabular}
\caption[chromosome]{a chromosome.}
\label{table_chromosome}
\end{center}
\end{table}

It can be seen that in general, for the property buying decision, the genome
matches the strategy described previously. In general, the gene values for
buying a location are higher when a player already owns one of the properties of
a group compared to when two other players own properties in the group. Although
it is not shown in Figure~\ref{figure-genome999}, the gene values for buying
when one opponent owns are generally higher than when two opponents own a property in
the group. Tables for other genomes can be found in Appendix~\ref{sec:appendixB}

When compared to the strategy list from~\ref{m_gamestrategies}, there might
appear to be an inconsistency in the genome. For example, the strategy says
always buy a property if no one else owns it. However, only the gene value for
Park Place approaches a probability of 1.0 which is implied by the strategy.
This can easily be explained by the fact that if the player declines a property
with probability \(p_{decline} = 1-p_{buy}\), the probability of subsequently
buying the property is higher. This is because when a player declines a
property, it is then auctioned to any player including the declining player, and
the declining player uses the same chromosome to make the bid decision
independently of the buy decision. So the probability of deciding to buy a
property is
\begin{equation*}
1-p_{decline} \cdot p_{decline}
\end{equation*}
For example, even when the probability of a positive buy decision in the
chromosome is as low as 0.6, the probability of declining is 0.4, and the
probability of attempting to buy the property is 0.84. the probability of
actually obtaining the property is slightly lower, however, since it is
dependent on winning the auction.

\section{SGA Player} \label{6_SGA}

After evolving the population of RGA Players, the simulation was conducted again
using SGA Players. SGA players are those players with a binary string
chromosome.

The fitness distribution in the first generation shows the same general
similarity to the binomial distribution (Figure~\ref{figure-sga_gen0}), although
there appears to be a bit of skewness.

\begin{figure}[htp]
\centerline{\includegraphics[width=0.75\columnwidth]{Figures/sga_gen0.png}}
\caption[SGA Fitness Generation 0]{This chart shows the actual fitness
distribution for the first generation of an SGA population.}
\label{figure-sga_gen0}
\end{figure}

Figure~\ref{figure-sga_gen289} shows the fitness distribution at generation 289
and the skew is definitely present. It is unclear why this would have occurred.

\begin{figure}[htp]
\centerline{\includegraphics[width=0.75\columnwidth]{Figures/sga_gen289.png}}
\caption[SGA Fitness Generation 289]{This chart shows the actual fitness
distribution for the generation 289 of an SGA population.}
\label{figure-sga_gen289}
\end{figure}

\section{TGA Player} \label{6_TGS}

Finally, the same simulation was conducted using TGA players. The results for
the TGA player were essentially the same as for the RGA player.

The fitness distributions for the initial populations had the same shape as a
binomial distribution, with similar minimums and maximums for the initial
generation as for the RGA chromosomes. Over the 1000 generation simulations, the
fitness distribution converged towards the mean fitness. The minimum, maximum,
mean, and median were essentially the same as the RGA simulation, and the player
genomes were also very similar. This is unsurprising, since the fitness seems to
depend mostly on buying property (in which the RGA and TGA genomes are
identical), and not on getting out of jail which was the difference between the
two genomes.

\section{Validation} \label{6_Validation}

As with other research that used competitive fitness functions, validations of
the various populations were performed to test if the populations were actually
getting better.

\subsection{Intrapopulation validation}

For every population, the best player in the last generation was played against
the best player in generation 250, 500, and 750. 30 games were played with the
same set of 4 best players. If the population was actually evolving better
players, we expect that on average, the best player in the last generation will
win more games then the other players; and the best player in the 250th
generation should lose more games on average. The results are shown in
Figure~\ref{figure-intrapopulation}.

\begin{figure}[htp]
\centerline{\includegraphics[width=0.75\columnwidth]{Figures/figureTBD.png}}
\caption[Intrapopulation validation]{The results of playing the best player in
the last generation against the best players in previous generations.}
\label{figure-intrapopulation}
\end{figure}

\subsection{Interpopulation validation I}

Next, the best players from the last generation of similar populations that used
different fitness evaluators were played against each other. This was done to
see which, if any, fitness evaluator produced better players. Because of the
highly random nature of the game and based on the previous work comparing
competitive fitness functions, we expected that the TOURNAMENT fitness evaluator
would produce less fit players than at least some of the other fitness
evaluators.

ACTUAL RESULTS TBD. Figure~\ref{figure-interpopulation1} shows the results of
these competitions.

\begin{figure}[htp]
\centerline{\includegraphics[width=0.75\columnwidth]{Figures/figureTBD.png}}
\caption[Validation - Comparing Fitness Evaluators]{The results of playing the
best player in the last generation against the best players in previous
generations.}
\label{figure-interpopulation1}
\end{figure}

\subsection{Interpopulation validation II}

Based on the results of the evolutionary phase of this study, a set of RGA
players was selected and competed againt human players to validate the
evolutionary results. The results of these competitions is presented here.

The intra- and iter-population analysis showed that the best results came from
populations with high population size, many games per generation, and from the
latest generations of the evolution. For those reasons, players for this
validation were selected from the RGA-1024-100 population. 

 The best players from the last generation of various
populations that used the same fitness evaluators were also played against each other. This was done
to see if the populations that were smaller could produce comparable
players as the larger populations. As seen in Chapter~\ref{chap:background},
many previous studies used small populations to produce their results, whereas
the study that this thesis most closely resembles used a poulation of 1000
individuals. For this work, the populations with 1024 individuals took the
longest to evolve. Being able to get comparable results with smaller populations
would be a more efficient use of resources.

ACTUAL RESULTS TBD. Figure~\ref{figure-interpopulation2} shows the results of
these competitions.

\begin{figure}[htp]
\centerline{\includegraphics[width=0.75\columnwidth]{Figures/figureTBD.png}}
\caption[Validation - Comparing population sizes]{The results of playing the
best player in the last generation against the best players in previous
generations.}
\label{figure-interpopulation2}
\end{figure}

\section{Human versus RGA competition}
      % Table generated by Excel2LaTeX from sheet 'Pivot table'
      \begin{table}[htbp]
        \centering
        \caption[Human versus RGA results, initial]{Results of Human versus RGA
        player competitions, with initial trading algorithm}
          \begin{tabular}{r|rrr}
          \toprule
                 & Results &        &  \\
          \midrule
          Players & Average finish & Average networth & Count of games
          \\ \\
          \multicolumn{1}{l|}{gen250} & 2.8    & 1170.1 & 85.0 \\
          \hline
          \multicolumn{1}{l|}{player0040.dat} & 2.6    & 2207.0 & 26.0 \\
          \multicolumn{1}{l|}{player0191.dat} & 2.5    & 765.2  & 15.0 \\
          \multicolumn{1}{l|}{player0478.dat} & 3.0    & 416.2  & 22.0 \\
          \multicolumn{1}{l|}{player0752.dat} & 3.0    & 974.6  & 22.0 \\ \\
          \multicolumn{1}{l|}{gen500} & 2.8    & 1137.4 & 78.0 \\
          \hline
          \multicolumn{1}{l|}{player0028.dat} & 2.8    & 902.8  & 25.0 \\
          \multicolumn{1}{l|}{player0445.dat} & 2.9    & 1291.4 & 21.0 \\
          \multicolumn{1}{l|}{player0913.dat} & 2.7    & 404.6  & 19.0 \\
          \multicolumn{1}{l|}{player0923.dat} & 2.8    & 2410.8 & 13.0 \\ \\
          \multicolumn{1}{l|}{gen750} & 2.7    & 1367.1 & 92.0 \\
          \hline
          \multicolumn{1}{l|}{player0453.dat} & 2.9    & 332.1  & 28.0 \\
          \multicolumn{1}{l|}{player0474.dat} & 2.6    & 1910.8 & 28.0 \\
          \multicolumn{1}{l|}{player0524.dat} & 2.6    & 1195.1 & 18.0 \\
          \multicolumn{1}{l|}{player0750.dat} & 2.6    & 2303.2 & 18.0 \\ \\
          \multicolumn{1}{l|}{gen999} & 2.9    & 954.7  & 72.0 \\
          \hline
          \multicolumn{1}{l|}{player0101.dat} & 3.0    & 762.1  & 18.0 \\
          \multicolumn{1}{l|}{player0423.dat} & 3.1    & 730.9  & 14.0 \\
          \multicolumn{1}{l|}{player0667.dat} & 2.7    & 1645.2 & 13.0 \\
          \multicolumn{1}{l|}{player0928.dat} & 3.0    & 866.6  & 27.0 \\ \\
          \multicolumn{1}{l|}{human} & 1.6    & 7484.0 & 109.0 \\
          \hline
          \multicolumn{1}{l|}{A} & 1.6    & 7926.6 & 19.0 \\
          \multicolumn{1}{l|}{B} & 1.3    & 9141.6 & 24.0 \\
          \multicolumn{1}{l|}{C} & 2.3    & 5443.4 & 8.0 \\
          \multicolumn{1}{l|}{E} & 1.9    & 6311.3 & 12.0 \\
          \multicolumn{1}{l|}{G} & 2.3    & 3030.3 & 3.0 \\
          \multicolumn{1}{l|}{H} & 1.6    & 7371.2 & 13.0 \\
          \multicolumn{1}{l|}{S} & 1.5    & 7644.7 & 23.0 \\
          \multicolumn{1}{l|}{K} & 1.0    & 11473.0 & 3.0 \\
          \multicolumn{1}{l|}{V} & 2.5    & 5654.5 & 2.0 \\
          \multicolumn{1}{l|}{L} & 2.0    & 0.0    & 1.0 \\
          \multicolumn{1}{l|}{T} & 3.0    & 0.0    & 1.0 \\ \\
          \multicolumn{1}{l|}{Grand Total} & 2.5    & 2748.7 & 436.0 \\
          \bottomrule
          \end{tabular}%
        \label{tab:addlabel}%
      \end{table}%
